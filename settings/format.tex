% 设置字体主题(参考曾祥东的latex-talk.tex进行设置)
%the font themes that currently come with beamer (i.e. files in $TEXMF/latex/beamer/themes/font) are:
% default
% serif
% professionalfonts
% structurebold
% structureitalicserif
% structuresmallcapsserif
%\usefonttheme{serif,professionalfonts}
%\usefonttheme{structurebold}
% 加载字体(注意需要安装相应字体,在此使用的都是免费字体)
% 附字体下载链接
% - [iosevka](https://github.com/be5invis/Iosevka/releases)
% - [Libertinus](https://github.com/alif-type/libertinus/releases)
% - [sarasa-gothic/mono](https://github.com/be5invis/Sarasa-Gothic/releases)
% - [SourceHanSerif](https://github.com/adobe-fonts/source-han-serif/releases)
% - [SourceHanSans](https://github.com/adobe-fonts/source-han-sans/releases)
% 设置字体(参考曾祥东的latex-talk.tex进行设置)
%\usefonttheme{serif,professionalfonts}
\usefonttheme{professionalfonts}
% 加载字体
% 加载字体
\setmainfont{LibertinusSerif}[% 英文字体
  Extension      = .otf,
  UprightFont    = *-Regular,
  BoldFont       = *-Bold,
  ItalicFont     = *-Italic,
  BoldItalicFont = *-BoldItalic,
  Scale          = 1.0]
\setmonofont{Iosevka Term}[% 英文等宽字体,主要用于代码排版
  Scale=1.00,
  BoldFont        = * Heavy,
  UprightFont     = * Semibold,
  BoldFont        = * Extrabold,
  ItalicFont      = * Light,
  BoldItalicFont  = * Medium,
  RawFeature      = +fwid]
\setCJKmainfont{Source Han Serif SC}[ % 中文衬线字体,思源宋体
  UprightFont     = * SemiBold,
  BoldFont        = * Heavy,
  ItalicFont      = * Light,
  BoldItalicFont  = * Medium,
  RawFeature      = +fwid]
\setCJKsansfont{Source Han Sans SC}[ % 中文无衬线字体,思源宋体
  UprightFont     = * Medium,
  BoldFont        = * Heavy,
  ItalicFont      = * Light,
  BoldItalicFont  = * Normal,
  RawFeature      = +fwid]  
\setCJKmonofont{Sarasa Mono SC}[% 中文等宽字体,Sarasa Mono SC
  UprightFont     = * Semibold,
  BoldFont        = * Bold,
  ItalicFont      = * Italic,
  BoldItalicFont  = * Bold Italic,
  RawFeature      = +fwid]
% 字体命令
\newCJKfontfamily\fangsong{FangSong}
\newCJKfontfamily\songti{Source Han Serif SC}
\newCJKfontfamily\heiti{Source Han Sans SC}
\newCJKfontfamily\kaishu{Adobe Kaiti Std} 

% 各类标题字体设置
\setbeamerfont{title}{size=\huge, series=\bfseries}
\setbeamerfont*{subtitle}{size=\large}%shape=\itshape
\setbeamerfont{section title}{size=\Large}%, series=\bfseries
\setbeamerfont{frametitle}{size=\large, series=\bfseries}%
\setbeamerfont{caption}{size=\footnotesize, series=\bfseries}
\setbeamerfont{footnote}{size=\tiny}
%\setbeamerfont{alerted text}{series=\bfseries}
\setbeamertemplate{itemize/enumerate subbody begin}{\footnotesize}
\setbeamertemplate{caption}{\parbox{\textwidth}{\centering\insertcaption}\par}
\setbeamertemplate{bibliography item}[text]

% 如果需要更改主题中不同元素的颜色,请取消相应注释并编辑为喜欢的颜色
% 分割条和边栏颜色:
%\setbeamercolor{NWSUAFsidebar}{fg=red!20,bg=red}
%\setbeamercolor{sidebar}{bg=red!20}
% 结构元素颜色:
\setbeamercolor{structure}{fg=red}
% 帧标题颜色:
%\setbeamercolor{frametitle}{fg=blue!25}
% 正文文本背景色:
%\setbeamercolor{normal text}{bg=gray!10}
% ... 如果需要更改更多的参数,请参考 beamer 用户手册.
% \setbeamertemplate{blocks}[default]

\definecolor{Descitem}{RGB}{0, 0, 139}

\definecolor{StdTitle}{RGB}{26, 33, 141}
\definecolor{StdBody}{RGB}{213,24,0}

\definecolor{AlTitle}{RGB}{255, 190, 190}
\definecolor{AlBody}{RGB}{213,24,0}

\definecolor{ExTitle}{RGB}{201, 217, 217}
\definecolor{ExBody}{RGB}{213,24,0}  
  
% Standard block
\setbeamercolor{block title}{fg = Descitem, bg = StdTitle!15!white}
\setbeamercolor{block body}{bg = StdBody!5!white}
% Alert block
\setbeamercolor{block title alerted}{bg = AlTitle}
\setbeamercolor{block body alerted}{bg = AlBody!5!white}
% Example block
\setbeamercolor{block title example}{bg = ExTitle}
\setbeamercolor{block body example}{bg = ExBody!5!white}

\setbeamerfont{block title}{size=\scriptsize}  
\setbeamertemplate{blocks}[rounded][shadow=true]

% 设置脚注字号
\setbeamerfont{footnote}{size=\zihao{7}}

%%%%%%%% User Specified Commands %%%%%%%%
 \setbeamercolor{alerted text}{fg=red}
 %\setbeamercolor{alerted text}{fg=red!2!green!35!blue}
\newenvironment{boxalertenv}{\begin{altenv}%
  {\usebeamertemplate{alerted text begin}\usebeamercolor[fg]
    {alerted text}\usebeamerfont{alerted text}\colorbox{bg}}
  {\usebeamertemplate{alerted text end}}{\color{.}}{}}{\end{altenv}}
\newcommand<>{\boxalert}[1]{{%
  \begin{boxalertenv}#2{#1}\end{boxalertenv}%
}}

\def\hilite<#1>{%
  \temporal<#1>{\color{gray}}{\color{red!2!green!35!blue}}%
    {\color{blue!25}}}

% colored hyperlinks
\newcommand{\chref}[2]{%
  \href{#1}{{\usebeamercolor[bg]{Aalborg}#2}}
}

% 定义绘制内存的命令(基于bytefield宏包)
%facilitates the creation of memory maps. Start
% address at the bottom, end address at the top.  syntax:
% \memsection{end address}{start address}{height in lines}{text in box}
\newcommand{\memsection}[5][]{
  \bytefieldsetup{bitheight=#4\baselineskip} % define the height of the memsection
  \bitbox[]{10}{\raggedleft \texttt{#1}\hspace{1em} \texttt{#2}% print end address
    \\ \vspace{#4\baselineskip} \vspace{-2\baselineskip}
    \vspace{-#4pt} % do some spacing
    \texttt{#3}% print start address
  }~ \bitbox{6}{#5} % print box with caption
}

% \newcommand{\memsection}[4]{
%   \bytefieldsetup{bitheight=#3\baselineskip} % define the height of the memsection
%   \bitbox[]{6}{ \texttt{#1} % print end address
%     \\ \vspace{#3\baselineskip} \vspace{-2\baselineskip}
%     \vspace{-#3pt} % do some spacing
%     \texttt{#2} % print start address
%   } \bitbox{6}{#4} % print box with caption
% }

% 原始版本
% % facilitates the creation of memory maps. Start address at the bottom, end address at the top.
% % Addresses will be print with a leading '0x' and in upper case.
% % syntax: \memsection{end address}{start address}{height in lines}{text in box}
% \newcommand{\memsection}[4]{
%       \bytefieldsetup{bitheight=#3\baselineskip}      % define the height of the memsection
%       \bitbox[]{8}{
%               \texttt{0x\uppercase{#1}}        % print end address
%               \\ \vspace{#3\baselineskip} \vspace{-2\baselineskip} \vspace{-#3pt} % do some spacing
%               \texttt{0x\uppercase{#2}} % print start address
%       }
%       \bitbox{16}{#4} % print box with caption
% }

% 定义颜色
% ==================================================
\definecolor{mypink}{rgb}{.99,.91,.95}
\definecolor{mycyan}{cmyk}{.3,0,0,0}
\definecolor{listinggray}{gray}{0.9}
\definecolor{lbcolor}{rgb}{0.9,0.9,0.9}
\definecolor{Blue}{rgb}{1.,0.75,0.8}
\newcommand{\cppfillcolor}{yellow!20}

% 参考文献格式
% \bibliographystyle{plain}

% ==================================================

% 代码显示模式设置
% ==================================================
\usemintedstyle{default}  %codeblocks模式
% 通用设置
\setminted{fontsize=\tiny, breaklines=true, breakautoindent=false}
% 行间代码自定义环境
\newminted{cpp}{bgcolor=\cppfillcolor,autogobble,frame=lines}
\newminted[cpptt]{cpp}{autogobble,mathescape,frame=lines,escapeinside=||}
%\newminted[asmtt]{nasm}{bgcolor=\cppfillcolor,autogobble,mathescape,fontsize=\tiny,frame=lines,escapeinside=||}
\newminted[cppttnobg]{cpp}{autogobble,mathescape,frame=lines,escapeinside=||}

% % 不同字体大小的行内代码自定义命令
\newmintinline{cpp}{fontsize=\normalsize}
\newmintinline[cppinlinett]{cpp}{fontsize=\normalsize, escapeinside=||}
\newmintinline[cppintt]{cpp}{fontsize=\normalsize, escapeinside=||}
\newmintinline[cppintttny]{cpp}{fontsize=\tiny, escapeinside=||}
\newmintinline[cppinttscr]{cpp}{fontsize=\scriptsize, escapeinside=||}
\newmintinline[cppinttfts]{cpp}{fontsize=\footnotesize, escapeinside=||}
\newmintinline[cppinttlrg]{cpp}{fontsize=\large, escapeinside=||}

% 文件载入代码自定义环境
% 注意,此处不可以使用autogobble命令,否则无法正常载入代码文件
\newmintedfile{cpp}{frame=lines}%linenos=true,
\newmintedfile[cppfilett]{cpp}{frame=lines,escapeinside=||}%linenos=true,
\newmintedfile[cppfiletikz]{cpp}{frame=lines,escapeinside=||}%linenos=true,
\newmintedfile[cppfilenobg]{cpp}{mathescape,frame=lines}
\newmintedfile[cppfilettnobg]{cpp}{mathescape,frame=lines,escapeinside=||}

\newenvironment{mytabbing}[1][]
  {\par#1\tabbing}
  {\endtabbing\par}

% 为需要的章节定义该命令  
\ifcase\chno
% 第0章  
\or
% 第1章
\or
% 第2章
\or
% 第3章
%====================用tcolorbox定义一个代码盒子==============================
\tcbuselibrary{skins, xparse, minted}
\usetikzlibrary{shapes.geometric}
%------------------------------------------------------------------------------------
% tcolorbox lang代码样式定义
%------------------------------------------------------------------------------------
\tcbset{%
  lang/.style={%
    drop shadow,%    
    arc=0mm,%
    right=0pt,%
    top=0pt,%
    bottom=0pt,%
    left=0pt,%
    enhanced jigsaw,
    %colframe=tcbcolback!60!black,%
    colframe=blue!50!black,%
    colback=yellow!20,%tcbcolback!30!white,%
    colbacktitle=tcbcolback!5!yellow!10!white,%
    fonttitle=\scriptsize\bfseries,%
    coltitle=black,%
    attach boxed title to top left={%
      xshift=0.6cm,%
      yshift*=0.5mm-\tcboxedtitleheight%
    },%
    %varwidth boxed title*=-3cm,%
    boxed title style={%
      frame code={%
        \path[fill=blue!55!black]([yshift=-1mm,xshift=-1mm]frame.north west)%
        arc[start angle=0,end angle=180,radius=1mm]([yshift=-1mm,xshift=1mm]frame.north east)%
        arc[start angle=180,end angle=0,radius=1mm];%
        \path[left color=tcbcolback!60!black,right color=tcbcolback!60!black,
        middle color=tcbcolback!80!black]([xshift=-2mm]frame.north west)%
        --([xshift=2mm]frame.north east)[rounded corners=1.0mm]%
        --([xshift=1mm,yshift=-1mm]frame.north east)%
        --(frame.south east)%
        --(frame.south west)%
        --([xshift=-1mm,yshift=-1mm]frame.north west)[sharp corners]%
        --cycle;%
      },%
      interior engine=empty,% 
      size=small,
      top=-1mm,
      bottom=-1mm,
    },%
  }% 
}% end tcolorbox lang style

\DeclareTCBListing{cpptcb}{ O{} m }{%
  listing engine=minted,%
  minted style=default,%
  minted options={%
    breaklines,%
    fontsize=\tiny,%
    escapeinside=#1,
  },%
  listing only,%
  lang,%
  title={#2},%
  minted language=cpp%
}% end codebox

\or
% 第4章
\or
% 第5章
\or
% 第6章
\or
% 第7章
\or
% 第8章
\or
% 第9章
\or
% 第10章
\fi
%   ===========================================================

% 为了在\scalebox中使用minted,先定义盒子
% \newsavebox{\cppbox}

% TikZ宏包扩展
%\usetikzlibrary{graphdrawing}
\usetikzlibrary{graphs}
\usetikzlibrary{mindmap,trees}
% Here we change the style for all concepts: (Stefan K.)
\tikzset{every concept/.style={minimum size=1cm, text width=1.5cm}}
\usetikzlibrary{shapes,shapes.geometric,chains}
\usetikzlibrary{positioning}
\usetikzlibrary{calc}
\usetikzlibrary{arrows.meta}
\usetikzlibrary{decorations.pathreplacing}
\usetikzlibrary{tikzmark}
%\usetikzlibrary{graphs}
%\usegdlibrary{trees}
%% Two commands for representing content.

\newcommand\memcontent[2][2cm]{%%' 
  \begin{minipage}{#1}
    \centering
    #2
  \end{minipage}}

% Set up a few colours
\colorlet{lcfree}{green}
\colorlet{lcnorm}{blue}
\colorlet{lccong}{red}

% styles for flowcharts
\tikzstyle{block} = [rectangle, draw, text width=10em, text centered,rounded corners, minimum height=1.0em]

% 定义节点标注命令
% \newcommand\tikzmark[1]{%
%   \tikz[overlay,remember picture] \node[coordinate] (#1) {};%
% }

% 插图中的坐标单位
\setlength\unitlength{1mm}

% \setlength{\parindent}{2em}

%% 定义自动扩展垂直间距的命令\stretchon和\stretchoff
%% ==================================================
\def\itemsymbol{$\blacktriangleright$}
\let\svpar\par
\let\svitemize\itemize
\let\svenditemize\enditemize
\let\svitem\item
\let\svcenter\center
\let\svendcenter\endcenter
\let\svcolumn\column
\let\svendcolumn\endcolumn
\def\newitem{\renewcommand\item[1][\itemsymbol]{\vfill\svitem[##1]}}%
\def\newpar{\def\par{\svpar\vfill}}%
\newcommand\stretchon{%
  \newpar%
  \renewcommand\item[1][\itemsymbol]{\svitem[##1]\newitem}%
  \renewenvironment{itemize}%
    {\svitemize}{\svenditemize\newpar\par}%
  \renewenvironment{center}%
    {\svcenter\newpar}{\svendcenter\newpar}%
  \renewenvironment{column}[2]%
    {\svcolumn{##1}\setlength{\parskip}{\columnskip}##2}%
    {\svendcolumn\vspace{\columnskip}}%
}
\newcommand\stretchoff{%
  \let\par\svpar%
  \let\item\svitem%
  \let\itemize\svitemize%
  \let\enditemize\svenditemize%
  \let\center\svcenter%
  \let\endcenter\svendcenter%
  \let\column\svcolumn%
  \let\endcolumn\svendcolumn%
}

% 解决默认强调字体是italic,此时中文会用楷体代替,
% 在此设置为加粗,注意需要使用etoolbox宏包
\makeatletter
\let\origemph\emph
\newcommand*\emphfont{\normalfont\bfseries}
\DeclareTextFontCommand\@textemph{\emphfont}
\newcommand\textem[1]{%
  \ifdefstrequal{\f@series}{\bfdefault}
    {\@textemph{\CTEXunderline{#1}}}
    {\@textemph{#1}}%
}
\RenewDocumentCommand\emph{s o m}{%
  \IfBooleanTF{#1}
    {\textem{#3}}
    {\IfNoValueTF{#2}
      {\textem{#3}\index{#3}}
      {\textem{#3}\index{#2}}%
     }%
}
\makeatother
% ================================================

%% 签署春秋学期日期命令
\newcommand{\tomonth}{
  \the\year 年\the\month 月
}


\newcommand{\tomonthen}{
  \ifcase\the\month
  \or January%
  \or February%
  \or March%
  \or April%
  \or May%
  \or June%
  \or July%
  \or August%
  \or September%
  \or October%
  \or November%
  \or December%
  \fi, \the\year
}

\newcommand{\tosemester}{
  \the\year 年\ 
  \ifcase\the\month
  \or 秋%
  \or 春%
  \or 春%
  \or 春%
  \or 春%
  \or 春%
  \or 春%
  \or 夏%
  \or 秋%
  \or 秋%
  \or 秋%
  \or 秋%
  \fi 
}

\newcommand{\tosemesteren}{  
  \ifcase\the\month
  \or Autumn%
  \or Spring%
  \or Spring%
  \or Spring%
  \or Spring%
  \or Spring%
  \or Summer%
  \or Autumn%
  \or Autumn%
  \or Autumn%
  \or Autumn%
  \or Autumn%
  \fi, \the\year
}

\newlength\columnskip
\columnskip 0pt
%% ==================================================

%% 自定义相关的名称宏命令
%% ==================================================
%% \newcommand{\yourcommand}[参数个数]{内容}
% 西北农林科技大学各单位名称
\newcommand{\nwsuaf}{西北农林科技大学}
\newcommand{\cie}{信息工程学院}
\newcommand{\ca}{农学院}
\newcommand{\cpp}{植物保护学院}
\newcommand{\ch}{园艺学院}
\newcommand{\cast}{动物科技学院}
\newcommand{\cvm}{动物医学院}
\newcommand{\cf}{林学院}
\newcommand{\claa}{风景园林艺术学院}
\newcommand{\cnre}{资源环境学院}
\newcommand{\cwrae}{水利与建筑工程学院}
\newcommand{\cmee}{机械与电子工程学院}
\newcommand{\cfse}{食品科学与工程学院}
\newcommand{\ce}{葡萄酒学院}
\newcommand{\cls}{生命科学学院}
\newcommand{\cst}{理学院}
\newcommand{\ccp}{化学与药学院}
\newcommand{\cem}{经济管理学院}
\newcommand{\cm}{马克思主义学院}
\newcommand{\dfl}{外语系}
\newcommand{\iec}{创新实验学院}
\newcommand{\ci}{国际学院}
\newcommand{\dpe}{体育部}
\newcommand{\cvae}{成人教育}
\newcommand{\iswc}{水土保持研究所}

\newcommand{\cs}{计算机科学系}
% 定义引号命令
\newcommand{\qtmark}[1]{``#1''}

%叉号与对号,需要用到pifont宏包
\newcommand{\goodmark}{\textcolor{green!50!black}{\Pisymbol{pzd}{52}}}
\newcommand{\badmark}{\textcolor{red}{\Pisymbol{pzd}{56}}}

% 路径设置
% ==================================================
\graphicspath{{figure/}}%图片所在的目录
% ==================================================

% 为标题页指定一个 logo
\pgfdeclareimage[height=0.8cm]{titlepagelogo}{nwafulogo/h_bar}% 标题页
\titlegraphic{% 标题页底部
  \pgfuseimage{titlepagelogo}
%  \hspace{1cm}\pgfuseimage{titlepagelogo2}
}

% 每一个frame中要开始添加的命令,需要etoolbox宏包支持
% ==================================================
%\AtBeginEnvironment{frame}{\stretchon}
%\preto\frame{\stretchon}
%\BeforeBeginEnvironment{frame}{\stretchon}
% ==================================================

% 每一个frame中要结束添加的命令,需要etoolbox宏包支持
% ==================================================
%\AtEndEnvironment{frame}{\stretchoff}
%\appto\frame{\stretchoff}
%\AfterEndEnvironment{frame}{\stretchoff}
% ==================================================


% % 每一讲前面添加的帧
% % ==================================================
% \AtBeginLecture{
%   \begin{frame}{目录}{本讲主要内容}
%     %\Large
%     %\centering
%     %\insertlecture
%     \tableofcontents
%   \end{frame}
% }

%%% Local Variables: 
%%% mode: latex
%%% TeX-master: "../main.tex"
%%% End: 
